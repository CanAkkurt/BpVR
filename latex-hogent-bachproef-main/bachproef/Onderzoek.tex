\chapter{Het onderzoek}%
\label{ch:Het onderzoek}

\subsection{Inleiding}
Dit hoofdstuk presenteert een vergelijkende studie die de effectiviteit van de AI-gestuurde applicatie evalueert tegenover traditionele methoden voor het verkrijgen van informatie over culturele en historische sites. Deze methoden omvatten het gebruik van webbronnen, fysieke verkenning ter plaatse, en interactie met mensen in de omgeving.

\subsection{Methodologie voor Vergelijkende Studie}
De studie omvatte het verzamelen van gegevens van gebruikers die verschillende methoden gebruikten om informatie over dezelfde culturele bezienswaardigheden te verkrijgen. Elk van deze methoden werd beoordeeld op basis van criteria zoals toegankelijkheid, gebruikersvriendelijkheid, nauwkeurigheid van de informatie, en de algehele gebruikerstevredenheid.

\subsection{Datacollectie}
Data werden verzameld via enquêtes en directe observatie, waarbij gebruikers werden gevraagd hun ervaringen te documenteren met elke methode. Deelnemers waren een diverse groep die willekeurig werd toegewezen aan een van de methoden, met inachtneming van variabelen zoals leeftijd, achtergrond, en technische vaardigheid.

\subsection{Analyse van Resultaten}
De analyse van de verzamelde gegevens is cruciaal voor het bepalen van de effectiviteit van de verschillende methoden om informatie over culturele bezienswaardigheden te verkrijgen. Deze sectie beschrijft de statistische en kwalitatieve methoden die zijn gebruikt om de prestaties van de AI-gestuurde applicatie, webzoekopdrachten, fysieke verkenning en interactie met mensen in de omgeving te evalueren.

\subsubsection{Statistische Analyse}
Kwantitatieve gegevens, zoals gebruikerstevredenheidsscores, tijdsduur voor het verkrijgen van informatie, en de nauwkeurigheid van de verkregen informatie, werden verzameld en geanalyseerd met behulp van de volgende statistische methoden:
\begin{itemize}
    \item \textbf{ANOVA (Analysis of Variance)}: Gebruikt om te bepalen of er significante verschillen zijn in de gemiddelde tevredenheidsscores tussen de verschillende methoden. Dit helpt bij het identificeren welke methode als het meest bevredigend wordt ervaren door de gebruikers.
    \item \textbf{Chi-kwadraat testen}: Toegepast om te onderzoeken of verschillen in categorieën zoals gebruikersvoorkeuren statistisch significant zijn.
    \item \textbf{Correlatieanalyse}: Uitgevoerd om de relaties tussen de leeftijd en technische vaardigheid van gebruikers en hun voorkeuren voor bepaalde informatieverzamelingsmethoden te beoordelen.
\end{itemize}

\textbf{Resultaten van ANOVA voor Gebruikerstevredenheid}:
\begin{table}[H]
    \centering
    \begin{tabular}{|l|c|c|}
        \hline
        \textbf{Methode}           & \textbf{Gemiddelde Tevredenheidsscore} & \textbf{Standaarddeviatie} \\ \hline
        AI Applicatie              & 4.3                                   & 0.75                      \\ \hline
        Webzoekopdrachten           & 3.8                                   & 0.90                      \\ \hline
        Fysieke Verkenning          & 3.2                                   & 1.10                      \\ \hline
        Interactie met mensen       & 2.6                                   & 1.20                      \\ \hline
    \end{tabular}
    \caption{ANOVA-resultaten voor gebruikerstevredenheid}
    \label{tab:anova-satisfaction}
\end{table}

De ANOVA toonde significante verschillen in tevredenheidsscores tussen de methoden (F(3, 52) = 4.67, p < 0.01). Post-hoc analyse (Tukey's HSD) liet zien dat interactie met mensen significant lagere tevredenheidsscores had vergeleken met de andere methoden.

\textbf{Tijdsduur voor het Verkrijgen van Informatie}:
\begin{table}[H]
    \centering
    \begin{tabular}{|l|c|c|}
        \hline
        \textbf{Methode}           & \textbf{Gemiddelde Tijd (minuten)} & \textbf{Standaarddeviatie} \\ \hline
        AI Applicatie              & 6                                 & 2                          \\ \hline
        Webzoekopdrachten           & 12                                & 4                          \\ \hline
        Fysieke Verkenning          & 20                                & 7                          \\ \hline
        Interactie met mensen       & 10                                & 5                          \\ \hline
    \end{tabular}
    \caption{Gemiddelde tijdsduur voor het verkrijgen van informatie}
    \label{tab:time-spent}
\end{table}

\textbf{Resultaten van ANOVA voor Informatienauwkeurigheid}:
\begin{table}[H]
    \centering
    \begin{tabular}{|l|c|c|}
        \hline
        \textbf{Methode}           & \textbf{Gemiddelde Nauwkeurigheidsscore} & \textbf{Standaarddeviatie} \\ \hline
        AI Applicatie              & 4.2                                     & 0.80                      \\ \hline
        Webzoekopdrachten           & 3.5                                     & 1.00                      \\ \hline
        Fysieke Verkenning          & 2.8                                     & 1.20                      \\ \hline
        Interactie met mensen       & 2.4                                     & 1.30                      \\ \hline
    \end{tabular}
    \caption{ANOVA-resultaten voor informatienauwkeurigheid}
    \label{tab:anova-accuracy}
\end{table}

De ANOVA voor informatienauwkeurigheid toonde significante verschillen aan tussen de methoden (F(3, 52) = 5.21, p < 0.01).

\subsubsection{Kwalitatieve Analyse}
Naast statistische methoden werd er ook een diepgaande kwalitatieve analyse uitgevoerd op de verzamelde feedback en open reacties:

\begin{itemize}
    \item \textbf{Thema-analyse}: Gebruikt om terugkerende thema’s en patronen binnen de gebruikersfeedback te identificeren, wat inzicht geeft in de gebruikerservaringen en percepties van de effectiviteit van elke methode.
    \item \textbf{Inhoudsanalyse}: Toegepast om de diepte en relevantie van de informatie die door elke methode wordt verkregen te beoordelen, en hoe deze informatie bijdraagt aan de verrijking van de culturele ervaring.
\end{itemize}

\textbf{Samenvatting van Kwalitatieve Bevindingen}:

\begin{itemize}
    \item \textbf{AI Applicatie}: Gebruikers waardeerden het gemak en de snelheid van toegang tot informatie, hoewel er meldingen waren van incidenten waarbij de applicatie niet het juiste culturele object identificeerde, wat soms leidde tot verwarring.
    \item \textbf{Webzoekopdrachten}: Gebruikers waardeerden de beschikbaarheid van uitgebreide informatie online, maar vonden het tijdrovend om door verschillende bronnen te bladeren en de nauwkeurigheid te verifiëren.
    \item \textbf{Fysieke Verkenning}: Gebruikers genoten van de ervaring van zelf ontdekken, maar voelden dat ze vaak belangrijke contextuele informatie misten die door gidsen of technologie kon worden verstrekt.
    \item \textbf{Interactie met mensen}: Gebruikers waardeerden de persoonlijke verhalen en tips, maar ervoeren ook inconsistentie in de kwaliteit van de informatie en noemden dat er soms geen mensen beschikbaar waren om te helpen.
\end{itemize}

\subsubsection{Geïntegreerde Data-analyse}
De resultaten van zowel de statistische als de kwalitatieve analyses werden geïntegreerd om een holistisch beeld te krijgen van de effectiviteit van de verschillende methoden:

\begin{itemize}
    \item \textbf{AI Applicatie}: Scoorde hoog op gemak en snelheid, met gemiddelde tevredenheid en nauwkeurigheid. Gebruikers merkten op dat hoewel de applicatie snel informatie bood, het soms moeite had om het juiste culturele object te identificeren.
    \item \textbf{Webzoekopdrachten}: Bood uitgebreide informatie maar was tijdrovend en soms onnauwkeurig.
    \item \textbf{Fysieke Verkenning}: Werd gewaardeerd voor de ervaring, maar had een lage tevredenheid en nauwkeurigheid vanwege het gebrek aan gedetailleerde informatie.
    \item \textbf{Interactie met mensen}: Had de laagste tevredenheid en nauwkeurigheidsscores vanwege inconsistente informatie en de soms afwezigheid van behulpzame personen.
\end{itemize}


