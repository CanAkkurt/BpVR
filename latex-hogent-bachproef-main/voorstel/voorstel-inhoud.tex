%---------- Inleiding ---------------------------------------------------------

\section{Introductie}%
\label{sec:introductie}

Mijn bachelorscriptie zal gaan over het onderwerp "De implementatie van kunstmatige intelligentie (AI) voor het verrijken van de ervaring van toeristen tijdens het bezoek aan historische en culturele locaties". Dit onderwerp wordt geplaatst binnen de veel bredere context van technologische innovatie binnen de toeristenindustrie, waarbij specifiek wordt gekeken hoe AI de interacties tussen bezoekers en andere belanghebbenden op het niveau van attracties kan herdefiniëren.

De primaire doelgroep zou alle belanghebbenden in het veld moeten zijn: van beleidsmakers tot degenen die de managers van historische en culturele sites organiseren, van ontwikkelaars van bijbehorende technologieën tot uiteraard de toeristen zelf. Elk van hen — beleidsmakers, sitebeheerders, technologieontwikkelaars en toeristen — heeft een belang bij het onderwerp: de eersten om manieren bij de hand te hebben om de aantrekkelijkheid en toegankelijkheid van de locaties te verhogen; de tweede om nieuwe toepassingsgebieden voor hun producten te ontwikkelen; en de derde om te zoeken naar verrijkte en gepersonaliseerde ervaringen.

Het probleem dat dit onderzoek probeert te identificeren, is hoe de toepassing van AI specifiek zou kunnen helpen om de ervaring van toeristen die historische en culturele sites bezoeken te verbeteren. Dit leidt tot de hoofdonderzoeksvraag: "Op welke manier kan de toepassing van kunstmatige intelligentietechnologie binnen de erfgoed- en cultuursector bijdragen aan de verrijking van de bezoekerservaring aan historische en culturele site waarden, en hoe kan dat het beste worden geïmplementeerd?"

%---------- Stand van zaken ---------------------------------------------------

\section{State-of-the-art}%
\label{sec:state-of-the-art}


De huidige stand van zaken toont dat AI-technologieën, waaronder machine learning, natuurlijke taalverwerking, en computer vision, reeds worden geïntegreerd in verschillende facetten van de toeristische industrie. Dit varieert van gepersonaliseerde aanbevelingssystemen voor reizigers tot interactieve en augmented reality-toepassingen voor het verrijken van de bezoekerservaring bij culturele en historische sites. Belangrijke publicaties zoals die van \autocite{Zhou2018} benadrukken hoe AI de personalisatie van toeristische ervaringen kan verbeteren, terwijl \autocite{Umerov2023InnovativeDO} de potentie van AI in het verhogen van de toegankelijkheid en het educatieve aanbod van culturele sites verkent.

Desondanks blijven er open vragen en uitdagingen binnen het onderzoeksdomein. De effectieve integratie van AI in de toeristische sector vereist vaak nog verdere verfijning en adaptatie aan de specifieke context van historische en culturele sites. Er is ook een groeiende nood aan empirisch onderzoek dat de impact van AI-toepassingen op de bezoekerservaring kwantificeert.

Mijn onderzoek bouwt voort op deze bestaande literatuur en streeft ernaar om een lacune te identificeren: de directe toepasbaarheid en effectiviteit van AI binnen de niche van historische en culturele toerisme-ervaringen. In tegenstelling tot eerdere studies, die zich mogelijk concentreerden op brede AI-toepassingen in toerisme, zal dit onderzoek specifieke AI-gestuurde interventies en hun directe impact op de bezoeker bij culturele en historische sites analyseren.


% Voor literatuurverwijzingen zijn er twee belangrijke commando's:
% \autocite{KEY} => (Auteur, jaartal) Gebruik dit als de naam van de auteur
%   geen onderdeel is van de zin.
% \textcite{KEY} => Auteur (jaartal)  Gebruik dit als de auteursnaam wel een
%   functie heeft in de zin (bv. ``Uit onderzoek door Doll & Hill (1954) bleek
%   ...'')

%---------- Methodologie ------------------------------------------------------
\section{Methodologie}%
\label{sec:methodologie}

De eerste stap is een uitgebreide literatuurstudie van het huidige onderzoekslandschap, waarbij de toepassingen van AI in de context van toerisme, met name op historische en culturele locaties, worden onderzocht. Dit omvat het identificeren en analyseren van relevante literatuur, technologieën en best practices.
\begin{itemize}
    \item \textit{Doel:} Inzicht verkrijgen in bestaande AI-oplossingen en identificeren van onderzoeksgaten.
    \item \textit{Deliverable:} Literatuuroverzichtsrapport.
\end{itemize}

Ontwikkeling van de AI-toepassing:
In deze fase wordt een AI-prototype ontwikkeld dat de bezoekerservaring verbetert. Hier worden de technologie en methodologieën bepaald en ontwikkeld voor het prototype.
\begin{itemize}
    \item \textit{Doel:} Een functioneel AI-prototype creëren.
    \item \textit{Deliverable:} Werkend AI-prototype.
\end{itemize}

Validatie en Testen:
Het ontwikkelde prototype zal in een experimentele omgeving worden getest om de technische prestaties en gebruikerservaring te beoordelen, inclusief het verzamelen en analyseren van feedback en prestatiegegevens.
\begin{itemize}
    \item \textit{Doel:} De AI-toepassing evalueren en gebruikersfeedback verzamelen.
    \item \textit{Deliverable:} Testrapport en feedbackanalyse.
\end{itemize}

Analyse en Conclusie:
Deze laatste fase biedt een analyse van alle verzamelde gegevens om conclusies te trekken over de efficiëntie en het potentieel van de AI-toepassing. Op basis hiervan worden aanbevelingen voor verder onderzoek en toepassing gegeven.
\begin{itemize}
    \item \textit{Doel:} Conclusies en aanbevelingen formuleren op basis van de analyse.
    \item \textit{Deliverable:} Eindrapport met bevindingen en aanbevelingen.
\end{itemize}

%---------- Verwachte resultaten ----------------------------------------------
\section{Verwacht resultaat, conclusie}%
\label{sec:verwachte_resultaten}

Het doel van deze studie is dus om duidelijk te laten zien op welke mogelijke manieren kunstmatige intelligentie (AI) bezoeken aan culturele locaties aantrekkelijker, informatiever en plezieriger kan maken. We verwachten dat de interactie van alle bezoekers met de tentoonstellingen zal toenemen, dat ze meer zullen leren en met meer voldoening de locatie zullen verlaten, omdat AI deel uitmaakte van hun ervaring. Dit zal zeer waardevolle informatie zijn voor mensen die verantwoordelijk zijn voor de marketing van deze sites, aangezien het de praktische voordelen van het gebruik van AI illustreert. De resultaten, mochten deze tegen de verwachtingen in gaan, zullen onze ogen openen voor de redenen, wat leidt tot bevindingen en ons helpt om nog meer te begrijpen over het potentieel van AI in toerisme. Deze studie is bedoeld om duidelijk te maken hoe AI enorm kan bijdragen aan de ervaring van het verkennen van cultureel erfgoed.

