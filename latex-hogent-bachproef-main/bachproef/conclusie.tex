%%=============================================================================
%% Conclusie
%%=============================================================================

\chapter{Conclusie}%
\label{ch:conclusie}

% TODO: Trek een duidelijke conclusie, in de vorm van een antwoord op de
% onderzoeksvra(a)g(en). Wat was jouw bijdrage aan het onderzoeksdomein en
% hoe biedt dit meerwaarde aan het vakgebied/doelgroep? 
% Reflecteer kritisch over het resultaat. In Engelse teksten wordt deze sectie
% ``Discussion'' genoemd. Had je deze uitkomst verwacht? Zijn er zaken die nog
% niet duidelijk zijn?
% Heeft het onderzoek geleid tot nieuwe vragen die uitnodigen tot verder 
%onderzoek?

\section{Samenvatting van de Belangrijkste Bevindingen}
Het onderzoek toonde aan dat de AI-gestuurde applicatie een effectieve balans biedt tussen gemak, nauwkeurigheid en gebruikerstevredenheid in vergelijking met traditionele methoden zoals webzoekopdrachten, fysieke verkenning, en interactie met mensen in de omgeving.

\begin{itemize}
    \item \textbf{AI Applicatie}: Scoorde hoog op gemak en snelheid van informatievoorziening, maar had soms moeite met de nauwkeurige identificatie van culturele objecten.
    \item \textbf{Webzoekopdrachten}: Bodem uitgebreide informatie, maar vereiste meer tijd en inspanning van de gebruiker om de nauwkeurigheid te verifiëren.
    \item \textbf{Fysieke Verkenning}: Gebruikers waardeerden de ervaring, maar misten vaak gedetailleerde en contextuele informatie.
    \item \textbf{Interactie met mensen}: Deze methode had de laagste tevredenheids- en nauwkeurigheidsscores vanwege inconsistente informatie en de soms afwezigheid van behulpzame personen.
\end{itemize}

\section{Aanbevelingen voor Toekomstig Onderzoek}
Op basis van de bevindingen van deze studie worden de volgende aanbevelingen gedaan voor toekomstig onderzoek:

\begin{itemize}
    \item \textbf{Vergroten van de Nauwkeurigheid}: Onderzoek naar methoden om de nauwkeurigheid van objectherkenning te verbeteren zou moeten worden voortgezet, met name in het herkennen van minder bekende of complexe culturele objecten.
    \item \textbf{Uitgebreidere Gebruikerstests}: Toekomstig onderzoek zou moeten profiteren van een grotere en meer diverse steekproef van gebruikers om de bevindingen te valideren en inzicht te krijgen in de behoeften van verschillende demografische groepen.
    \item \textbf{Integratie van Multimodale Informatie}: Het combineren van visuele, tekstuele en mogelijk audio-informatie kan een vollediger en rijker informatieprofiel bieden, wat de toeristische ervaring verder kan verbeteren.
    \item \textbf{Lange Termijn Effecten}: Onderzoek naar de lange termijn effecten van het gebruik van AI-gestuurde applicaties op het leren en onthouden van culturele en historische informatie zou waardevol zijn.
\end{itemize}
\section{Slotgedachten}
Dit onderzoek heeft aangetoond dat de integratie van kunstmatige intelligentie in toeristische applicaties veel potentie heeft om de ervaring van toeristen te verbeteren door hen te voorzien van nauwkeurige en relevante informatie. Hoewel er nog uitdagingen zijn, zoals de verbetering van de nauwkeurigheid van objectherkenning, biedt de technologie een veelbelovende toekomst voor het verrijken van culturele en historische educatie. Toekomstige ontwikkelingen en onderzoek zullen ongetwijfeld bijdragen aan het verder verbeteren van deze technologieën en hun toepassingen.
