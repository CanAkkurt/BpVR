%%=============================================================================
%% Samenvatting
%%=============================================================================

% TODO: De "abstract" of samenvatting is een kernachtige (~ 1 blz. voor een
% thesis) synthese van het document.
%
% Een goede abstract biedt een kernachtig antwoord op volgende vragen:
%
% 1. Waarover gaat de bachelorproef?
% 2. Waarom heb je er over geschreven?
% 3. Hoe heb je het onderzoek uitgevoerd?
% 4. Wat waren de resultaten? Wat blijkt uit je onderzoek?
% 5. Wat betekenen je resultaten? Wat is de relevantie voor het werkveld?
%
% Daarom bestaat een abstract uit volgende componenten:
%
% - inleiding + kaderen thema
% - probleemstelling
% - (centrale) onderzoeksvraag
% - onderzoeksdoelstelling
% - methodologie
% - resultaten (beperk tot de belangrijkste, relevant voor de onderzoeksvraag)
% - conclusies, aanbevelingen, beperkingen
%
% LET OP! Een samenvatting is GEEN voorwoord!

%%---------- Nederlandse samenvatting -----------------------------------------
%
% TODO: Als je je bachelorproef in het Engels schrijft, moet je eerst een
% Nederlandse samenvatting invoegen. Haal daarvoor onderstaande code uit
% commentaar.
% Wie zijn bachelorproef in het Nederlands schrijft, kan dit negeren, de inhoud
% wordt niet in het document ingevoegd.

\IfLanguageName{english}{%
\selectlanguage{dutch}
\chapter*{Samenvatting}
\lipsum[1-4]
\selectlanguage{english}
}{}

%%---------- Samenvatting -----------------------------------------------------
% De samenvatting in de hoofdtaal van het document

\chapter*{\IfLanguageName{dutch}{Samenvatting}{Abstract}}

Deze bachelorproef, getiteld "Implementatie van Kunstmatige Intelligentie voor Verbeterde Toeristische Ervaringen bij Historische en Culturele Bezienswaardigheden", onderzoekt de mogelijkheden van kunstmatige intelligentie (AI) om toeristische ervaringen te verrijken door middel van een innovatieve applicatie. Het doel van deze studie is om te evalueren hoe AI-technologieën kunnen worden toegepast om gedetailleerde en nauwkeurige informatie te verstrekken over culturele objecten en locaties, met als uiteindelijke doel de toeristische ervaring te verbeteren.

De centrale onderzoeksvraag luidt: "Hoe effectief is de implementatie van kunstmatige intelligentie in het verbeteren van toeristische ervaringen bij historische en culturele bezienswaardigheden?" Deze vraag werd onderzocht door een proof of concept (PoC) applicatie te ontwikkelen die AI-gestuurde beeldherkenning en natuurlijke taalverwerking integreert.

De methodologie omvatte het ontwerpen en ontwikkelen van een Android-applicatie die foto's van culturele objecten kan analyseren. De applicatie maakt gebruik van de Google Vision API om objecten te herkennen en ChatGPT 3.5 om contextuele beschrijvingen te genereren. De gegenereerde informatie wordt vervolgens opgeslagen in een SQL-database. De effectiviteit van de applicatie werd getest door middel van een vergelijkende studie waarbij 14 gebruikers de AI-gestuurde methode vergeleken met traditionele methoden zoals het gebruik van gidsen, webzoekopdrachten en QR-codes.

De resultaten van het onderzoek toonden aan dat de AI-gestuurde applicatie in staat was om rijke en nauwkeurige informatie te bieden, wat de gebruikerservaring aanzienlijk verbeterde. Gebruikers waardeerden het gemak en de snelheid waarmee de informatie beschikbaar werd gesteld. Echter, de applicatie had soms moeite met de nauwkeurigheid van de objectherkenning, wat een aandachtspunt is voor toekomstige verbeteringen.

Conclusie: De implementatie van kunstmatige intelligentie in toeristische applicaties biedt aanzienlijke voordelen en heeft de potentie om de manier waarop toeristen informatie verkrijgen te transformeren. Dit onderzoek toont aan dat AI een waardevolle toevoeging kan zijn in de toeristische sector, mits de technologie verder wordt geoptimaliseerd. Toekomstig onderzoek zou zich kunnen richten op het verbeteren van de herkenningsnauwkeurigheid en het uitbreiden van de functionaliteiten van de applicatie om een nog rijkere gebruikerservaring te bieden.
