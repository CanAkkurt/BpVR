\chapter{\IfLanguageName{dutch}{Stand van zaken}{State of the art}}%
\label{ch:stand-van-zaken}

% Tip: Begin elk hoofdstuk met een paragraaf inleiding die beschrijft hoe
% dit hoofdstuk past binnen het geheel van de bachelorproef. Geef in het
% bijzonder aan wat de link is met het vorige en volgende hoofdstuk.

% Pas na deze inleidende paragraaf komt de eerste sectiehoofding.

Dit hoofdstuk bevat je literatuurstudie. De inhoud gaat verder op de inleiding, maar zal het onderwerp van de bachelorproef *diepgaand* uitspitten. De bedoeling is dat de lezer na lezing van dit hoofdstuk helemaal op de hoogte is van de huidige stand van zaken (state-of-the-art) in het onderzoeksdomein. Iemand die niet vertrouwd is met het onderwerp, weet nu voldoende om de rest van het verhaal te kunnen volgen, zonder dat die er nog andere informatie moet over opzoeken \autocite{Pollefliet2011}.

Je verwijst bij elke bewering die je doet, vakterm die je introduceert, enz.\ naar je bronnen. In \LaTeX{} kan dat met het commando \texttt{$\backslash${textcite\{\}}} of \texttt{$\backslash${autocite\{\}}}. Als argument van het commando geef je de ``sleutel'' van een ``record'' in een bibliografische databank in het Bib\LaTeX{}-formaat (een tekstbestand). Als je expliciet naar de auteur verwijst in de zin (narratieve referentie), gebruik je \texttt{$\backslash${}textcite\{\}}. Soms is de auteursnaam niet expliciet een onderdeel van de zin, dan gebruik je \texttt{$\backslash${}autocite\{\}} (referentie tussen haakjes). Dit gebruik je bv.~bij een citaat, of om in het bijschrift van een overgenomen afbeelding, broncode, tabel, enz. te verwijzen naar de bron. In de volgende paragraaf een voorbeeld van elk.

\textcite{Knuth1998} schreef een van de standaardwerken over sorteer- en zoekalgoritmen. Experten zijn het erover eens dat cloud computing een interessante opportuniteit vormen, zowel voor gebruikers als voor dienstverleners op vlak van informatietechnologie~\autocite{Creeger2009}.

Let er ook op: het \texttt{cite}-commando voor de punt, dus binnen de zin. Je verwijst meteen naar een bron in de eerste zin die erop gebaseerd is, dus niet pas op het einde van een paragraaf.

\section{AI}
\subsection{Definitie}

Artificial intelligence (AI) represents a broad field of technology where machines are designed to operate with a form of intelligence that mirrors human cognition. AI systems can learn, reason, perceive, infer, communicate, and make decisions to achieve specific objectives. These systems have evolved from basic reactive machines without memory, like chess-playing computers, to sophisticated systems that can drive cars, diagnose medical conditions, or manage financial investments.

AI's classification into types, such as reactive machines, limited memory, theory of mind, and self-awareness, illustrates its progression and future potential. While we currently leverage reactive and limited memory AI in various applications like gaming, autonomous vehicles, and predictive analytics, the more advanced stages like theory of mind and self-awareness represent the frontier of AI research and development.

The implications of AI span across multiple sectors, including healthcare, where it accelerates diagnosis and treatment processes; business, where it enhances customer service and operational efficiency; education, by providing personalized learning experiences; and manufacturing, where it boosts production speed and quality.

However, AI's advancement also introduces challenges, such as ethical considerations, bias, job displacement, and security concerns. As AI systems become more autonomous, ensuring they make decisions that align with human values and ethics becomes paramount. Additionally, there's a need for frameworks to manage AI's societal impacts, particularly in terms of employment and privacy.

\subsection{Object recognition}
Artificial intelligence (AI) and object recognition technology have significantly impacted various sectors, including the recreation sector. Deep learning-based object detection models are essential in identifying and labeling different objects within an image or video, which can be highly beneficial for enhancing user experiences in recreational activities and facilities.

The basic structure of these AI models involves an encoder and decoder mechanism. The encoder processes the image to extract features, while the decoder predicts bounding boxes and labels for each detected object. While simpler models rely on regression to predict object locations directly, more advanced systems like region proposal networks and single shot detectors (SSDs) offer greater flexibility and accuracy by identifying potential object locations and refining predictions through additional network layers.

In the context of the recreation sector, such technologies could be applied in various innovative ways. For example, they could assist in managing and monitoring public spaces, enhancing safety through surveillance, facilitating crowd counting, or even improving visitor engagement by offering interactive experiences through mobile applications or augmented reality.

Moreover, adapting these AI models for edge deployment, such as in mobile or IoT devices used in recreational settings, involves specific considerations to ensure they function efficiently despite resource constraints. Techniques like network pruning, quantization, and adjusting input-output sizes are critical for optimizing model performance without compromising the user experience.

For a more detailed understanding and technical insights into object recognition and its applications, you might want to explore the discussions presented by Edge AI and Vision Alliance and Fritz AI:

Edge AI and Vision Alliance provides an overview of object recognition with MATLAB and its applications across various industries, which could be extrapolated to recreational scenarios.
Fritz AI delves into the intricacies of different deep learning-based approaches to object detection, outlining their structure, benefits, and limitations, as well as considerations for edge deployment.
By leveraging such AI technologies, the recreation sector can significantly enhance operational efficiency, safety, and user engagement, paving the way for more innovative and interactive recreational experiences.

\subsection{Gebruik in de recreatie sector}

The integration of Artificial Intelligence (AI) in the recreation sector, particularly in parks and recreation management, is rapidly evolving. AI applications like ChatGPT are being used by park and recreation professionals to create more efficient administrative processes, such as drafting program announcements and developing emergency action plans. These AI tools serve as valuable assets for ideation and handling writing-intensive tasks, acting as an "infinitely patient intern" for staff, thereby optimizing operational efficiency and enhancing community engagement \autocite{ParksRecAI2023}.

In the broader scope of the entertainment industry, AI's role extends to various aspects, including content recommendation, creation, post-production, and audience analytics. For instance, AI algorithms analyze user data to offer personalized content recommendations on streaming platforms, enhancing viewer engagement. AI is also employed in content creation, assisting in generating scripts and storylines. Additionally, AI tools are leveraged in post-production tasks like video editing and visual effects, streamlining workflows and elevating the quality of the final product. Audience analytics through AI provides invaluable insights into viewer preferences and behavior, enabling informed decision-making regarding content strategies and targeted advertising \autocite{WebisoftEntAI}.