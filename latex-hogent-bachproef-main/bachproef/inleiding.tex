%%=============================================================================
%% Inleiding
%%=============================================================================

\chapter{\IfLanguageName{dutch}{Inleiding}{Introduction}}%
\label{ch:inleiding}

De moderne toeristische sector staat voor de uitdaging om technologisch onderlegde bezoekers te voorzien van rijke, nauwkeurige en relevante informatie over culturele en historische bezienswaardigheden. Traditionele methoden zoals het raadplegen van gidsen, het doorzoeken van webbronnen of het fysiek verkennen van een locatie zijn vaak tijdrovend en kunnen de bezoekerservaring belemmeren door het gebrek aan direct beschikbare en nauwkeurige informatie. Dit onderzoek richt zich op de ontwikkeling en evaluatie van een AI-gestuurde applicatie die deze kloof kan overbruggen door gebruik te maken van geavanceerde beeldherkenning en natuurlijke taalverwerkingstechnologieën.

Het gebruik van kunstmatige intelligentie (AI) in toerisme biedt ongekende mogelijkheden om de bezoekerservaring te verbeteren. Door het integreren van beeldherkenning via de Vision API en contentgeneratie via ChatGPT 3.5, kan een mobiele applicatie toeristen in real-time voorzien van gedetailleerde en accurate informatie over de objecten en locaties die zij bezoeken. Dit onderzoek streeft ernaar om de effectiviteit van een dergelijke applicatie te beoordelen in vergelijking met traditionele methoden zoals webzoekopdrachten, fysieke verkenning en interactie met lokale gidsen of bewoners.

\section{\IfLanguageName{dutch}{Probleemstelling}{Problem Statement}}%
\label{sec:probleemstelling}

Er is een groeiende behoefte aan innovatieve oplossingen die toeristen kunnen helpen bij het verkrijgen van betrouwbare en diepgaande informatie over culturele en historische bezienswaardigheden. Traditionele methoden kunnen vaak niet voldoen aan de behoefte aan direct beschikbare en nauwkeurige informatie. Deze bachelorproef stelt de focus op het onderzoeken van de effectiviteit van een AI-gestuurde mobiele applicatie om deze uitdaging aan te pakken. De doelgroep van dit onderzoek bestaat uit toeristen en culturele instellingen die streven naar verbeterde educatieve en informatieve ervaringen.

\section{\IfLanguageName{dutch}{Onderzoeksvraag}{Research Question}}%
\label{sec:onderzoeksvraag}

De centrale onderzoeksvraag van deze bachelorproef is: "Hoe effectief is een AI-gestuurde applicatie in het verbeteren van de toeristische ervaring door het verstrekken van informatie over culturele en historische bezienswaardigheden in vergelijking met traditionele methoden?" Deze vraag wordt verder opgesplitst in de volgende deelvragen:
\begin{itemize}
    \item Hoe nauwkeurig en relevant is de door de AI-gestuurde applicatie verstrekte informatie?
    \item Hoe gebruiksvriendelijk is de applicatie in vergelijking met andere methoden?
    \item Wat zijn de percepties van gebruikers over de applicatie in termen van toegevoegde waarde en educatieve inhoud?
\end{itemize}

\section{\IfLanguageName{dutch}{Onderzoeksdoelstelling}{Research Objective}}%
\label{sec:onderzoeksdoelstelling}

Het verwachte resultaat van deze bachelorproef is een proof-of-concept van een AI-gestuurde applicatie die in staat is om toeristen te voorzien van nauwkeurige en relevante informatie over culturele en historische bezienswaardigheden. De criteria voor succes omvatten:
\begin{itemize}
    \item Ontwikkeling van een functionerende AI-gestuurde mobiele applicatie.
    \item Integratie van de Vision API voor beeldherkenning en ChatGPT 3.5 voor natuurlijke taalverwerking.
    \item Uitvoering van een vergelijkende studie om de effectiviteit van de applicatie te evalueren ten opzichte van traditionele methoden.
    \item Analyse van gebruikersfeedback en gebruikerstevredenheid.
\end{itemize}

\section{\IfLanguageName{dutch}{Opzet van deze bachelorproef}{Structure of this bachelor thesis}}%
\label{sec:opzet-bachelorproef}

De rest van deze bachelorproef is als volgt opgebouwd:

In Hoofdstuk~\ref{ch:stand-van-zaken} wordt een overzicht gegeven van de stand van zaken binnen het onderzoeksdomein, op basis van een literatuurstudie. Dit omvat een bespreking van bestaande technologieën en methoden voor het verstrekken van toeristische informatie, evenals een beoordeling van de huidige toepassingen van AI in deze context.

In Hoofdstuk~\ref{ch:methodologie} wordt de methodologie toegelicht en worden de gebruikte onderzoekstechnieken besproken om een antwoord te kunnen formuleren op de onderzoeksvragen. Dit hoofdstuk beschrijft in detail de ontwikkeling van de AI-gestuurde applicatie, de integratie van de Vision API en ChatGPT 3.5, en de implementatie van de Java-backend en SQL-database.

In Hoofdstuk~\ref{ch:PoC} wordt de proof-of-concept (PoC) van de AI-gestuurde applicatie gepresenteerd. Dit hoofdstuk bevat een gedetailleerde beschrijving van de architectuur van de applicatie, de datastroom en de integratie van de verschillende componenten. Het biedt ook een overzicht van de uitdagingen en oplossingen die tijdens de ontwikkeling zijn tegengekomen.

In Hoofdstuk~\ref{ch:Het onderzoek} wordt een vergelijkende studie gepresenteerd die de effectiviteit van de AI-gestuurde applicatie evalueert tegenover traditionele methoden voor het verkrijgen van informatie over culturele en historische sites. De methoden omvatten het gebruik van webbronnen, fysieke verkenning ter plaatse, en interactie met mensen in de omgeving. De analyse van de verzamelde gegevens, zowel statistisch als kwalitatief, wordt in detail beschreven.

In Hoofdstuk~\ref{ch:conclusie}, tenslotte, wordt de conclusie gegeven en een antwoord geformuleerd op de onderzoeksvragen. Daarbij wordt ook een aanzet gegeven voor toekomstig onderzoek binnen dit domein. Dit hoofdstuk vat de belangrijkste bevindingen samen, bespreekt de implicaties van deze bevindingen en biedt aanbevelingen voor verdere ontwikkeling en onderzoek.

