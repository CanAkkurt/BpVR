\chapter{\IfLanguageName{dutch}{Stand van zaken}{State of the art}}%
\label{ch:stand-van-zaken}


Dit hoofdstuk biedt een overzicht van de huidige stand van zaken op het gebied van kunstmatige intelligentie (AI), met een specifieke focus op objectherkenningstechnologieën en hun toepassingen in verschillende sectoren, waaronder de recreatiesector. AI speelt een steeds grotere rol in het verbeteren van operationele efficiëntie en gebruikerservaringen. In het bijzonder worden de diverse API's voor objectherkenning besproken, met als doel de meest geschikte API te identificeren voor integratie in de Proof of Concept (PoC).



\section{AI}
\subsection{Definitie}

De artikelen van Britannica en McKinsey bieden een grondig overzicht van kunstmatige intelligentie (AI) en de diverse toepassingen ervan in verschillende sectoren. De evolutie van AI wordt benadrukt door zijn vermogen om menselijke intelligentieprocessen te simuleren, waardoor vooruitgang mogelijk is in uiteenlopende gebieden, variërend van gezondheidszorg tot financiën en verder \autocite{mckinseyAI}.

AI wordt breed ingedeeld in vier typen op basis van functionaliteit en complexiteit: reactieve machines, machines met beperkt geheugen, theorie van de geest en zelfbewuste machines. De eenvoudigste, reactieve machines, werken zonder enige voorkennis en reageren alleen op huidige input. Machines met beperkt geheugen kunnen gebruikmaken van eerdere ervaringen om beslissingen te nemen, een cruciale functie voor toepassingen zoals zelfrijdende auto's. De meer geavanceerde en hypothetische categorieën, theorie van de geest en zelfbewuste machines, zouden machines inhouden die emoties en gedachten van anderen begrijpen en zelfbewustzijn bezitten, wat respectievelijk aanzienlijke sprongen naar echte AI vertegenwoordigt \autocite{courseraAI}.

De toepassing van AI in predictief onderhoud en logistieke optimalisatie illustreert hoe AI de operationele efficiëntie aanzienlijk kan verbeteren, kosten kan verlagen en de service-uitkomsten kan verbeteren. Zo kan AI-gestuurd predictief onderhoud bijvoorbeeld apparatuurstoringen voorzien voordat ze zich voordoen, waardoor proactief ingrijpen mogelijk is. In de logistiek optimaliseert AI routering en levering, wat de efficiëntie en klanttevredenheid ten goede komt \autocite{mckinseyAI}.

Generatieve AI, een ander opkomend vakgebied, staat op het punt om de creatie van inhoud over verschillende domeinen te revolutioneren, waarbij het helpt bij taken zoals het genereren van marketingteksten, de analyse van juridische documenten en zelfs het beoordelen van softwarecode. Hoewel het potentieel van AI enorm is, gaat het ook gepaard met ethische overwegingen, waarbij zorgvuldige overweging van inzet en implicaties noodzakelijk is \autocite{mckinseyAI}.
\subsection{Object recognition}
Kunstmatige intelligentie (AI) en objectherkenningstechnologieën hebben aanzienlijk hun aanwezigheid gemarkeerd in verschillende domeinen, waaronder de recreatiesector. Op diep leren gebaseerde objectdetectiemodellen zijn nu integraal in staat om verschillende elementen binnen afbeeldingen of video's te onderscheiden en te labelen, wat een belangrijke rol speelt bij het verbeteren van de gebruikerservaring in recreatieve locaties en activiteiten.

Deze AI-modellen bevatten een encoder om afbeeldingen te analyseren en kenmerken te identificeren, en een decoder om de locaties en labels van gedetecteerde objecten vast te stellen. Voorbij de basismodellen die regressie gebruiken voor directe locatievoorspellingen, verfijnen meer geavanceerde frameworks zoals region proposal networks en single-shot detectors (SSD's) de nauwkeurigheid door waarschijnlijke objectgebieden aan te wijzen en deze voorspellingen te verfijnen via ingewikkelde netwerklagen.

Binnen de recreatie-industrie bieden deze technologieën nieuwe toepassingen, zoals het verbeteren van veiligheid met surveillancetechnieken, het helpen bij crowdmanagement, of het stimuleren van betrokkenheid door interactieve digitale ervaringen mogelijk te maken, met behulp van augmented reality of mobiele apps.

Het aanpassen van AI voor edge-implementatie - optimalisatie voor gebruik in mobiele of IoT-apparaten binnen recreatieve contexten - vereist strategieën om efficiëntie te behouden onder beperkte middelen. Benaderingen zoals netwerkpruning en kwantisatie worden essentieel, om ervoor te zorgen dat de modelprestaties robuust blijven zonder de gebruikerservaring te verslechteren.



\subsection{Gebruik in de recreatie sector}

De integratie van Kunstmatige Intelligentie (AI) in de recreatiesector, met name in parken en recreatiemanagement, evolueert snel. AI-toepassingen zoals ChatGPT worden gebruikt door park- en recreatieprofessionals om efficiëntere administratieve processen te creëren, zoals het opstellen van programma-aankondigingen en het ontwikkelen van noodactieplannen. Deze AI-tools dienen als waardevolle hulpmiddelen voor ideeënontwikkeling en het omgaan met schrijfintensieve taken, en fungeren als een "oneindig geduldige stagiair" voor het personeel, waardoor operationele efficiëntie wordt geoptimaliseerd en de betrokkenheid van de gemeenschap wordt vergroot \autocite{ParksRecAI2023}.

In een breder perspectief binnen de entertainmentindustrie strekt de rol van AI zich uit tot verschillende aspecten, waaronder contentaanbeveling, creatie, postproductie en publieksanalyse. Zo analyseren AI-algoritmen gebruikersgegevens om gepersonaliseerde contentaanbevelingen te doen op streamingplatforms, waardoor de betrokkenheid van kijkers wordt vergroot. AI wordt ook ingezet bij contentcreatie, waarbij het helpt bij het genereren van scripts en verhaallijnen. Daarnaast worden AI-tools gebruikt bij postproductietaken zoals videobewerking en visuele effecten, waardoor workflows worden gestroomlijnd en de kwaliteit van het eindproduct wordt verhoogd. Publieksanalyse via AI biedt onschatbare inzichten in de voorkeuren en gedragingen van kijkers, waardoor geïnformeerde besluitvorming mogelijk is met betrekking tot contentstrategieën en gerichte advertenties \autocite{WebisoftEntAI}.

\section{Andere technologieën}
Deze sectie bespreekt hoe men met andere technologieën informatie kan vinden over culturele bezienswaardigheden. Elke technologie biedt unieke voordelen en toepassingen, die hieronder in detail worden besproken.

\subsection{Gebruik van QR-codes}
QR-codes (Quick Response codes) zijn een populaire technologie die informatie snel en eenvoudig toegankelijk maakt via smartphones. Musea, historische locaties en toeristische attracties gebruiken QR-codes om bezoekers toegang te geven tot gedetailleerde informatie, video's en interactieve inhoud. Bezoekers kunnen eenvoudig een QR-code scannen met hun smartphone om meer te leren over het object of de locatie die zij bekijken.

QR-codes bieden verschillende voordelen:
\begin{itemize}
    \item \textbf{Toegankelijkheid:} QR-codes zijn gemakkelijk te gebruiken en vereisen alleen een smartphone met een camera en een QR-code lezer.
    \item \textbf{Interactiviteit:} QR-codes kunnen gebruikers doorverwijzen naar multimedia-inhoud zoals video's, audiofragmenten en interactieve kaarten \autocite{lin2013qr}.
    \item \textbf{Kostenbesparing:} Het plaatsen van QR-codes is kostenefficiënt en kan eenvoudig worden bijgewerkt zonder fysieke aanpassingen aan de locatie.
    \item \textbf{Milieuvriendelijkheid:} Door digitale informatie te verstrekken, wordt het gebruik van papier verminderd, wat beter is voor het milieu \autocite{neal2014implementing}.
\end{itemize}

\subsection{Augmented Reality (AR)}
Augmented Reality (AR) wordt steeds vaker gebruikt in toerisme en educatie om bezoekers een verrijkte ervaring te bieden. Door middel van AR kunnen gebruikers hun smartphones of AR-brillen richten op een object of locatie en direct toegang krijgen tot interactieve en visuele informatie, zoals historische beelden, 3D-modellen en audiogidsen. AR-toepassingen kunnen bijvoorbeeld virtuele reconstructies van historische locaties laten zien, wat bijdraagt aan een dieper begrip van de geschiedenis en cultuur.

Voordelen van AR:
\begin{itemize}
    \item \textbf{Visuele Rijkdom:} AR kan complexe en rijke visuele informatie bieden die verder gaat dan traditionele tekstuele uitleg \autocite{azuma1997survey}.
    \item \textbf{Interactie:} Gebruikers kunnen interactief omgaan met de inhoud, zoals het manipuleren van 3D-modellen en het verkennen van virtuele omgevingen \autocite{billinghurst2015towards}.
    \item \textbf{Educatieve Waarde:} AR kan educatieve ervaringen verrijken door het bieden van dynamische en meeslepende leermiddelen \autocite{wu2013current}.
    \item \textbf{Betrokkenheid:} AR verhoogt de betrokkenheid van bezoekers door hen een unieke en boeiende ervaring te bieden \autocite{tom2018exploring}.
\end{itemize}

\subsection{Mobiele Applicaties}
Er zijn talloze mobiele applicaties ontwikkeld die toeristen helpen informatie te vinden over culturele en historische bezienswaardigheden. Applicaties zoals Google Arts \& Culture bieden uitgebreide informatie, virtuele rondleidingen en interactieve kaarten van musea en erfgoedsites wereldwijd. Dergelijke apps maken gebruik van geavanceerde technologieën zoals GPS, AI en databases met culturele informatie om gebruikers te voorzien van rijke en gepersonaliseerde ervaringen.

Kenmerken van mobiele applicaties:
\begin{itemize}
    \item \textbf{Toegankelijkheid:} Applicaties zijn overal beschikbaar en kunnen worden gedownload op smartphones en tablets \autocite{morrison2012mobile}.
    \item \textbf{Personalisatie:} Apps kunnen gepersonaliseerde ervaringen bieden op basis van gebruikersvoorkeuren en locatiegegevens \autocite{kim2013empirical}.
    \item \textbf{Breed Informatieaanbod:} Applicaties kunnen uitgebreide informatie bieden, waaronder tekst, afbeeldingen, video's en audiogidsen \autocite{wa2015analysis}.
    \item \textbf{Interactiviteit:} Gebruikers kunnen interactief omgaan met de inhoud en virtuele rondleidingen volgen \autocite{jung2015role}.
\end{itemize}

\subsection{Webzoekmachines en Online Encyclopedieën}
Webzoekmachines zoals Google en online encyclopedieën zoals Wikipedia zijn een van de meest gebruikte bronnen voor het verkrijgen van informatie over culturele en historische bezienswaardigheden. Gebruikers kunnen snel zoeken naar specifieke locaties en objecten en toegang krijgen tot uitgebreide artikelen, afbeeldingen en verwijzingen naar aanvullende bronnen. Deze platforms worden continu bijgewerkt en bieden vaak een breed scala aan informatie uit diverse perspectieven.

Voordelen van webzoekmachines en online encyclopedieën:
\begin{itemize}
    \item \textbf{Toegankelijkheid:} Deze bronnen zijn gemakkelijk toegankelijk via internet en bieden een schat aan informatie \autocite{mcclanahan2014wikipedia}.
    \item \textbf{Actualiteit:} Informatie wordt regelmatig bijgewerkt om nauwkeurig en actueel te blijven \autocite{giles2005internet}.
    \item \textbf{Breed Informatieaanbod:} Gebruikers kunnen een breed scala aan informatie vinden, van basisfeiten tot gedetailleerde analyses \autocite{okay2020information}.
    \item \textbf{Bronvermeldingen:} Artikelen bevatten vaak verwijzingen naar betrouwbare bronnen voor verder onderzoek \autocite{okoli2014wikipedia}.
\end{itemize}

\subsection{Social Media en User-Generated Content}
Social media platforms zoals Instagram, Facebook en TripAdvisor spelen een steeds belangrijkere rol in hoe mensen informatie vinden over culturele bezienswaardigheden. Gebruikers delen hun ervaringen, foto's en beoordelingen, wat waardevolle inzichten kan bieden voor andere toeristen. Daarnaast kunnen officiële pagina's van musea en toeristische organisaties actuele informatie, evenementen en interactieve inhoud delen met hun volgers.

Voordelen van social media en user-generated content:
\begin{itemize}
    \item \textbf{Echtheid:} Informatie en beoordelingen van echte gebruikers bieden authentieke inzichten en ervaringen \autocite{liu2015influence}.
    \item \textbf{Actualiteit:} Social media platforms bieden real-time updates en feedback van gebruikers \autocite{kwok2013exploring}.
    \item \textbf{Betrokkenheid:} Gebruikers kunnen direct communiceren met andere bezoekers en organisaties \autocite{schivinski2015influence}.
    \item \textbf{Visualisatie:} Foto's en video's van gebruikers geven een visuele indruk van bezienswaardigheden \autocite{munar2011social}.
\end{itemize}

\subsection{Digitale Gidsen en Audiotours}
Digitale gidsen en audiotours zijn handige hulpmiddelen voor toeristen die gedetailleerde informatie willen over bezienswaardigheden terwijl ze deze verkennen. Deze gidsen kunnen worden gedownload op smartphones en bieden vaak gesproken uitleg, interactieve kaarten en aanvullende multimedia-inhoud. Ze zijn bijzonder nuttig in situaties waar traditionele gidsen niet beschikbaar zijn of waar bezoekers op hun eigen tempo willen verkennen.

Voordelen van digitale gidsen en audiotours:
\begin{itemize}
    \item \textbf{Flexibiliteit:} Bezoekers kunnen op hun eigen tempo en volgens hun eigen schema de tour volgen \autocite{leue2015social}.
    \item \textbf{Interactiviteit:} Digitale gidsen bieden interactieve kaarten en multimedia-inhoud om de ervaring te verrijken \autocite{lee2016understanding}.
    \item \textbf{Toegankelijkheid:} Audiotours zijn toegankelijk voor mensen met visuele beperkingen \autocite{lee2016role}.
    \item \textbf{Kostenbesparing:} Digitale gidsen zijn vaak goedkoper dan het inhuren van een persoonlijke gids \autocite{king2009disney}.
\end{itemize}
\section{Object Recognition APIs in AI}

Objectherkenning is een cruciaal gebied in AI, waarbij verschillende API's aanzienlijk bijdragen aan de vooruitgang op dit gebied. In deze sectie worden enkele belangrijke API's en hun bijdragen besproken. Het doel is om de meest geschikte API te identificeren die in de Proof of Concept (PoC) verwerkt kan worden.


\subsubsection{Google Cloud Vision API}
Google Cloud Vision API biedt krachtige beeldanalysefuncties, waaronder objectherkenning, gezichtsdetectie, en tekstherkenning (OCR). Het maakt gebruik van geavanceerde machine learning-modellen om objecten in afbeeldingen te identificeren en te labelen. Gebruikers kunnen eenvoudig afbeeldingen uploaden en via API-aanroepen gedetailleerde analyses ontvangen. Deze API is bijzonder nuttig voor toepassingen zoals geautomatiseerde media-analyse en contentmoderatie\autocite{google2021vision}.

\subsubsection{Amazon Rekognition}
Amazon Rekognition is een beeld- en videobeeldanalyse-service die wordt aangeboden door AWS. Het kan objecten, mensen, tekst, scènes en activiteiten detecteren en herkennen, evenals ongepaste inhoud identificeren. Rekognition wordt vaak gebruikt voor beveiliging en bewaking, media- en entertainmenttoepassingen, en in marketing voor klantbetrokkenheid\autocite{amazonRekognition}.

\subsubsection{Microsoft Azure Computer Vision API}
Microsoft Azure Computer Vision API biedt geavanceerde mogelijkheden voor beeldanalyse, waaronder objectherkenning, tekstextractie (OCR), en merkdetectie. Deze API maakt gebruik van deep learning-modellen om rijke informatie uit afbeeldingen te halen. Het wordt veel gebruikt in sectoren zoals detailhandel, gezondheidszorg en productie om processen te automatiseren en inzichten te verkrijgen\autocite{microsoftAzureVision}.

\subsubsection{IBM Watson Visual Recognition}
IBM Watson Visual Recognition API maakt gebruik van krachtige AI-modellen om inhoud in visuele gegevens te analyseren. Het kan objecten, gezichten en scènes identificeren en classificeren, en biedt uitgebreide mogelijkheden voor het trainen van aangepaste modellen. Dit maakt het geschikt voor een breed scala aan toepassingen, van productidentificatie tot beveiligingsbewaking\autocite{ibmVisualRecognition}.

\subsubsection{Clarifai}
Clarifai biedt een uitgebreide suite van beeld- en videoherkenningstechnologieën, aangedreven door AI. De API kan objecten, gezichten, kleuren en concepten in afbeeldingen identificeren. Clarifai is bekend om zijn gebruiksgemak en de mogelijkheid om aangepaste modellen te trainen en te implementeren, wat het ideaal maakt voor maatwerkoplossingen in verschillende industrieën\autocite{clarifai}.

\subsubsection{DeepAI Image Recognition API}
DeepAI Image Recognition API biedt krachtige en flexibele beeldherkenningstechnologieën die kunnen worden geïntegreerd in verschillende toepassingen. Het kan objecten, scènes en gezichten in afbeeldingen herkennen en biedt een eenvoudig te gebruiken REST API. DeepAI wordt vaak gebruikt in onderzoeksprojecten, kunstmatige intelligentietoepassingen en softwareontwikkeling\autocite{deepai}.

\subsubsection{OpenCV AI Kit (OAK)}
OpenCV AI Kit (OAK) is een open-source hardware- en softwareplatform voor AI-gestuurde beeldverwerking en computer vision. OAK ondersteunt real-time objectdetectie, tracking en dieptewaarneming. Het is bijzonder nuttig voor embedded systemen en IoT-toepassingen waar rekenkracht en nauwkeurigheid van cruciaal belang zijn\autocite{opencv}.

\subsubsection{Alibaba Cloud Vision API}
Alibaba Cloud Vision API biedt uitgebreide beeldanalysefuncties, waaronder objectherkenning, gezichtsherkenning, en tekstanalyse. Deze API wordt ondersteund door krachtige AI-modellen en wordt gebruikt in diverse toepassingen, zoals e-commerce, beveiliging en smart cities. Alibaba Cloud Vision API is bekend om zijn schaalbaarheid en betrouwbaarheid\autocite{alibaba}.

\section{Webdetectie}
Webdetectie is een belangrijk onderdeel van beeldherkenningstechnologieën waarbij AI-modellen worden gebruikt om referenties naar specifieke afbeeldingen of objecten op het internet te vinden. Dit proces maakt gebruik van geavanceerde algoritmen om visuele kenmerken te analyseren en te matchen met een uitgebreide database van online beelden en bronnen. Een voorbeeld van een krachtige webdetectie-API is de Google Cloud Vision API.

De Google Cloud Vision API biedt uitgebreide mogelijkheden voor beeldanalyse, waaronder webdetectie. Deze functie kan objecten, teksten en andere visuele kenmerken in afbeeldingen identificeren en koppelen aan relevante online bronnen. Hierdoor kunnen gebruikers niet alleen de aard van de objecten begrijpen, maar ook contextuele informatie verkrijgen door het vinden van gerelateerde webpagina's en afbeeldingen.

\subsection{Hoe werkt Webdetectie?}
Webdetectie werkt door gebruik te maken van machine learning-modellen die zijn getraind op enorme hoeveelheden beelddata. Deze modellen analyseren de ingevoerde afbeelding, extraheren visuele kenmerken en zoeken vervolgens naar overeenkomsten in een grote online database. Het proces kan worden onderverdeeld in verschillende stappen:

\paragraph{1. Voorverwerking van de Afbeelding:}
Voordat een afbeelding kan worden geanalyseerd, wordt deze vaak voorverwerkt om de kwaliteit te optimaliseren. Dit kan inhouden dat de afbeelding wordt bijgesneden, de resolutie wordt aangepast, of ruis wordt verminderd. Deze stappen helpen om de nauwkeurigheid van de herkenning te verbeteren.

\paragraph{2. Kenmerkextractie:}
Tijdens de kenmerkextractie identificeert het model specifieke visuele kenmerken van de afbeelding, zoals kleuren, vormen, texturen en randen. Deze kenmerken worden omgezet in numerieke representaties die door het model kunnen worden geanalyseerd.

\paragraph{3. Matching en Vergelijking:}
De geëxtraheerde kenmerken worden vervolgens vergeleken met een uitgebreide database van beelden en visuele data. Het model zoekt naar overeenkomsten tussen de kenmerken van de ingevoerde afbeelding en de kenmerken in de database. Dit gebeurt door het berekenen van de gelijkenis tussen de numerieke representaties.

\paragraph{4. Identificatie van Webentiteiten:}
Wanneer overeenkomsten worden gevonden, identificeert het model de bijbehorende webentiteiten. Dit zijn vaak labels of beschrijvingen die aangeven wat er in de afbeelding is te zien. De webentiteiten worden gerangschikt op basis van hun relevantie en waarschijnlijkheid.

\paragraph{5. Genereren van Zoekresultaten:}
Op basis van de geïdentificeerde webentiteiten genereert de webdetectie-API een lijst van relevante zoekresultaten. Deze resultaten kunnen links bevatten naar webpagina's die de afbeelding of soortgelijke afbeeldingen bevatten. Dit stelt gebruikers in staat om meer contextuele informatie te vinden en verder onderzoek te doen naar het onderwerp van de afbeelding.

\subsection{Toepassingen van Webdetectie}
Webdetectie biedt tal van praktische toepassingen in verschillende domeinen:

\paragraph{Productherkenning:}
Webdetectie kan worden gebruikt om producten in afbeeldingen te identificeren en gebruikers door te verwijzen naar online winkels waar deze producten beschikbaar zijn. Dit is nuttig voor e-commerce platforms die visuele zoekfuncties willen integreren.

\paragraph{Auteursrechtbeheer:}
Door webdetectie kunnen auteursrechthouders controleren waar hun afbeeldingen online worden gebruikt. Dit helpt bij het opsporen van ongeautoriseerd gebruik en het beschermen van intellectuele eigendommen.

\paragraph{Culturele en Historische Onderzoek:}
In het kader van cultureel erfgoed kunnen onderzoekers afbeeldingen analyseren en relevante historische informatie vinden door vergelijkbare beelden online te identificeren. Dit vergemakkelijkt het verzamelen van informatie over kunstwerken, monumenten en historische locaties.

\paragraph{Content Moderatie:}
Webdetectie helpt bij het modereren van inhoud op online platforms door afbeeldingen te scannen op ongepaste of schadelijke inhoud. Door vergelijkbare beelden te identificeren, kunnen platforms sneller reageren op schendingen van hun beleid.

De Google Cloud Vision API is een voorbeeld van hoe webdetectie effectief kan worden ingezet om deze toepassingen te ondersteunen \autocite{googleVisionAPI}.


\section{ANOVA}
ANOVA, wat staat voor \textit{Analysis of Variance}, is een statistische methode die wordt gebruikt om de verschillen tussen de gemiddelden van drie of meer groepen te analyseren. Deze methode werd geïntroduceerd door Ronald A. Fisher in de jaren 1920 en is sindsdien een fundamenteel onderdeel van statistische analyses in vele onderzoeksgebieden.

ANOVA helpt bij het bepalen of er statistisch significante verschillen bestaan tussen de gemiddelden van drie of meer onafhankelijke groepen. Het hoofddoel van ANOVA is om te testen of de waargenomen verschillen in gemiddelden groter zijn dan wat door toeval zou worden verwacht.

ANOVA vergelijkt de variabiliteit tussen de groepen met de variabiliteit binnen de groepen. Deze vergelijking wordt uitgevoerd door de totale variantie van de data te splitsen in twee componenten:
\begin{itemize}
    \item \textbf{Tussen-groepen variabiliteit:} Variabiliteit veroorzaakt door het effect van de verschillende groepen.
    \item \textbf{Binnen-groepen variabiliteit:} Variabiliteit veroorzaakt door verschillen binnen dezelfde groep.
\end{itemize}

De basisprincipes van ANOVA kunnen als volgt worden samengevat:
\begin{enumerate}
    \item \textbf{Null Hypothese (H0):} Alle groepsgemiddelden zijn gelijk.
    \item \textbf{Alternatieve Hypothese (H1):} Ten minste één groepsgemiddelde is verschillend.
\end{enumerate}

De ANOVA-test geeft een F-statistiek, die de ratio is van de tussen-groepen variabiliteit tot de binnen-groepen variabiliteit. Een hoge F-waarde suggereert dat de variabiliteit tussen de groepen groot is ten opzichte van de variabiliteit binnen de groepen, wat kan wijzen op significante verschillen tussen de groepen.

Een uitgebreid overzicht van ANOVA en de bijbehorende technieken is te vinden in het boek \textit{Introduction to the Practice of Statistics} door David S. Moore, George P. McCabe, en Bruce A. Craig \autocite{moore2012introduction}.






