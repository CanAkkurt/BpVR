%%=============================================================================
%% Voorwoord
%%=============================================================================



\chapter*{\IfLanguageName{dutch}{Woord vooraf}{Preface}}%
\label{ch:voorwoord}

Voor u ligt mijn bachelorproef getiteld "Implementatie van Kunstmatige Intelligentie voor Verbeterde Toeristische Ervaringen bij Historische en Culturele Bezienswaardigheden". Deze scriptie markeert het einde van mijn studie Computerwetenschappen aan Hogent en vormt tegelijkertijd het begin van een nieuwe professionele reis.

Mijn dank gaat in de eerste plaats uit naar mijn promotor Chantal Teerlinck en co-promotor Franz Iacob, voor hun onschatbare begeleiding en voortdurende ondersteuning gedurende dit project. Hun inzichten en feedback hebben mij geholpen om deze scriptie naar een hoger niveau te tillen. Verder wil ik mijn familie en vrienden bedanken voor hun onvoorwaardelijke steun en aanmoediging tijdens mijn studieperiode.

Een speciaal woord van dank gaat uit naar de vrijwilligers die hun tijd en moeite hebben gestoken in het testen van de proof of concept. Hun waardevolle input en feedback waren cruciaal voor de ontwikkeling en verbetering van de applicatie.

Mijn passie voor technologie ontstond al op jonge leeftijd door een sterke nieuwsgierigheid naar hoe dingen werken en de mogelijkheden van innovatie. Dit leidde tot mijn studie Computerwetenschappen, waar ik een bijzondere interesse ontwikkelde voor kunstmatige intelligentie. Binnen dit domein raakte ik gefascineerd door de mogelijkheden van AI om menselijke ervaringen te verbeteren en te verrijken.

De inspiratie voor deze thesis kwam voort uit mijn liefde voor reizen en het verkennen van culturele locaties. Het idee dat AI kan bijdragen aan een rijkere en informatieve toeristische ervaring, sprak me enorm aan. Door de ontwikkeling van een applicatie die AI en beeldherkenning integreert, wil ik bezoekers van historische en culturele bezienswaardigheden een diepere en zinvollere ervaring bieden.

Dit onderzoek richt zich op het evalueren van verschillende methoden om informatie over culturele objecten en locaties te verkrijgen, met een specifieke focus op de toepassing van kunstmatige intelligentie. De resultaten van dit onderzoek bieden inzicht in de effectiviteit van AI-gestuurde applicaties in vergelijking met traditionele methoden zoals gidsen en online zoekopdrachten.

Ik hoop dat deze scriptie een goed inzicht geeft in de potentie van kunstmatige intelligentie om toeristische ervaringen te transformeren en inspireert tot verdere onderzoek en ontwikkeling op dit gebied.

Dank u voor uw interesse in mijn werk en veel leesplezier gewenst.



Kadir Can Akkurt

Bilzen 23/05/2024
\\